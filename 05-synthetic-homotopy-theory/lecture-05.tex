\documentclass[handout]{beamer}

\usepackage{graphicx}
\usepackage{tikz-cd}

\title{EPIT Lecture 5.5\\ The real projective spaces}
\author{Egbert Rijke}
\date{Friday, April 16th 2020}

\setbeamertemplate{caption}{\raggedright\insertcaption\par}

\mathchardef\usc="2D
\newcommand{\N}{\mathbb{N}}
\newcommand{\Z}{\mathbb{Z}}
\newcommand{\UU}{\mathcal{U}}
\newcommand{\brck}[1]{\|#1\|}
\newcommand{\Brck}[1]{\left\|#1\right\|}
\newcommand{\trunc}[2]{\|#2\|_{#1}}
\newcommand{\Trunc}[2]{\left\|#2\right\|_{#1}}
\newcommand{\unit}{\mathbf{1}}
\newcommand{\sphere}[1]{S^{#1}}
\newcommand{\isnull}{\mathsf{is\usc{}null}}
\newcommand{\htpy}{\sim}
\newcommand{\apbinary}{\mathsf{ap\usc{}bin}}
\newcommand{\Glob}{\mathsf{Glob}}
\newcommand{\typeGlob}{\mathsf{type}}
\newcommand{\relGlob}{\mathsf{rel}}
\newcommand{\homGlob}{\mathsf{hom}}
\newcommand{\maphomGlob}{\mathsf{map}}
\newcommand{\cgrhomGlob}{\mathsf{cgr}}
\newcommand{\bihomGlob}{\mathsf{bihom}}
\newcommand{\mapbihomGlob}{\mathsf{map}}
\newcommand{\cgrbihomGlob}{\mathsf{cgr}}
\newcommand{\ct}{\bullet}
\newcommand{\isconstant}[2]{\mathsf{is\usc{}}(#1,#2)\mathsf{\usc{}constant}}
\newcommand{\ap}{\mathsf{ap}}
\newcommand{\interchange}{\mathsf{interchange}}
\newcommand{\refl}{\mathsf{refl}}
\newcommand{\eh}{\mathsf{eckmann\usc{}hilton}}
\newcommand{\blank}{\mathord{\hspace{1pt}\text{--}\hspace{1pt}}}
\newcommand{\EM}{\mathsf{EM}}
\newcommand{\baseS}{\mathsf{base}}
\newcommand{\loopS}{\mathsf{loop}}
\newcommand{\apd}{\mathsf{apd}}
\newcommand{\tr}{\mathsf{tr}}
\newcommand{\idfunc}{\mathsf{id}}
\newcommand{\mulcircle}{\mu}
\newcommand{\basemulcircle}{\mathsf{base\usc{}mul}_{\sphere{1}}}
\newcommand{\loopmulcircle}{\mathsf{loop\usc{}mul}_{\sphere{1}}}
\newcommand{\htpyidcircle}{H}
\newcommand{\basehtpyidcircle}{\alpha}
\newcommand{\loophtpyidcircle}{\beta}
\newcommand{\invcircle}{\mathsf{inv}_{\sphere{1}}}
\newcommand{\evbase}{\mathsf{ev\usc{}base}}
\newcommand{\eqhtpy}{\mathsf{eq\usc{}htpy}}
\newcommand{\apply}[2]{#1(#2)}
\newcommand{\equiveq}{\mathsf{equiv\usc{}eq}}
\newcommand{\succZ}{\mathsf{succ}}
\newcommand{\bool}{\mathbf{2}}
\newcommand{\const}{\mathsf{const}}
\newcommand{\btrue}{\mathsf{true}}
\newcommand{\bfalse}{\mathsf{false}}
\newcommand{\ttt}{\mathsf{pt}}
\newcommand{\iscontr}{\mathsf{is\usc{}contr}}
\newcommand{\inr}{\mathsf{inr}}
\newcommand{\inl}{\mathsf{inl}}
\newcommand{\fib}{\mathsf{fib}}
\newcommand{\RP}[1]{\mathbb{R}\mathsf{P}^{#1}}

\setbeamertemplate{navigation symbols}{}
\setbeamertemplate{footline}[frame number]{}

\begin{document}

\begin{frame}
  \maketitle
\end{frame}

\begin{frame}
  \huge{Part 1. The type of 2-element types}
\end{frame}

\begin{frame}
  \begin{definition}
    The type of 2-element types is defined to be
    \begin{equation*}
      \UU_\bool :=\sum_{(X:\UU)}\brck{\bool\simeq X}
    \end{equation*}
  \end{definition}

  \begin{theorem}
    The type
    \begin{equation*}
      \sum_{(X:\UU_\bool)}X
    \end{equation*}
    of pointed 2-element types is contractible.
  \end{theorem}
\end{frame}

\begin{frame}
  \frametitle{Proof}
  The center of contraction is $(\bool,\eta(\refl),\btrue)$. Our goal is therefore to construct the contraction
  \begin{equation*}
    \prod_{(X:\UU)}\prod_{(p:\brck{\bool\simeq x})}\prod_{(x:X)}(\bool,\eta(\refl),\btrue)=(X,p,x).
  \end{equation*}
  We will show something slightly stronger, namely that
  \begin{equation*}
    \prod_{(X:\UU)}\prod_{(p:\brck{\bool\simeq x})}\prod_{(x:X)}\iscontr((\bool,\btrue)=(X,x)).
  \end{equation*}
  This is a proposition, so it suffices to show that
    \begin{equation*}
    \prod_{(X:\UU)}\prod_{(p:\bool\simeq x)}\prod_{(x:X)}\iscontr((\bool,\btrue)=(X,x)).
  \end{equation*}
  By univalence, it suffices to show that
  \begin{equation*}
    \prod_{(x:\bool)}\iscontr((\bool,\btrue)=(\bool,x)).
  \end{equation*}
\end{frame}

\begin{frame}
  \frametitle{Proof (cont).}
  Furthermore, by univalence it follows that the type $(X,x)=(Y,y)$ is equivalent to the type
  \begin{equation*}
    ((X,x)\simeq_\ast(Y,y)):=\sum_{(e:X\simeq Y)}e(x)=y
  \end{equation*}
  of base-point preserving equivalences from $X$ to $Y$.\\[\baselineskip]

  In other words, we have to show that for any $x:\bool$, there is a unique equivalence
  \begin{equation*}
    e:\bool\simeq\bool
  \end{equation*}
  such that $e(\btrue)=x$. This follows from the fact that the map
  \begin{equation*}
    ev:(\bool\simeq\bool)\to\bool
  \end{equation*}
  given by $e\mapsto e(\btrue)$ is an equivalence.\hfill$\square$
\end{frame}

\begin{frame}
  \begin{corollary}
    For any $2$-element type $X$, we have an equivalence
    \begin{equation*}
      (\bool=X)\simeq X.
    \end{equation*}
  \end{corollary}
  \begin{proof}
    By the fundamental theorem of identity types.
  \end{proof}
\end{frame}

\begin{frame}
  \begin{definition}
    A \textbf{line bundle} over a type $A$ is a map $A\to\UU_\bool$. 
  \end{definition}
  \begin{corollary}
    For any line bundle $B$ over $A$, the square
    \begin{equation*}
      \begin{tikzcd}[ampersand replacement=\&]
        \sum_{(x:A)}B(x) \arrow[d,swap,"\pi_1"] \arrow[r] \& \unit \arrow[d] \\
        A \arrow[r,swap,"B"] \& \UU_{\bool}
      \end{tikzcd}
    \end{equation*}
    is a pullback square.\footnote{We also say that $\unit\to\UU_{\bool}$ is the \textbf{universal line bundle}.}
  \end{corollary}
\end{frame}

\begin{frame}
  \huge{Part 2. The construction of the real projective spaces}
\end{frame}

\begin{frame}
  Classically, $\RP{n}$ is constructed out of $\sphere{n}$ by identifying antipodal pairs of points.\\[\baselineskip]\pause

  This implies that each $\RP{n}$ fits in a fiber sequence
  \begin{equation*}
    \begin{tikzcd}[ampersand replacement=\&]
      \bool \arrow[r] \& \sphere{n} \arrow[r] \& \RP{n}
    \end{tikzcd}
  \end{equation*}
  Our goal is to construct such fiber sequences in HoTT.
\end{frame}

\begin{frame}
  \frametitle{The descent theorem}
  \begin{equation*}
    \begin{tikzcd}[ampersand replacement=\&,row sep=small]
      \& S' \arrow[dl] \arrow[dr] \arrow[d] \\
      A' \arrow[d] \& S \arrow[dl] \arrow[dr] \& B' \arrow[dl,crossing over] \arrow[d] \\
      A \arrow[dr] \& X' \arrow[from=ul,crossing over] \arrow[d] \& B \arrow[dl] \\
      \& X
    \end{tikzcd}
  \end{equation*}
  Suppse the two squares in the back are pullback squares. The following are equivalent:
  \begin{enumerate}
  \item The two squares in the front are pullback squares.
  \item The square
    \begin{equation*}
      \begin{tikzcd}[ampersand replacement=\&,row sep=1.5em]
        A'\sqcup^{S'}B' \arrow[r] \arrow[d] \& X' \arrow[d] \\
        A\sqcup^{S}B \arrow[r] \& X
      \end{tikzcd}
    \end{equation*}
    is a pullback square.
  \end{enumerate}
\end{frame}

\begin{frame}
  \frametitle{The fiberwise join of two maps}
  Consider two maps $f:A\to X$ and $g:B\to X$. Then we can
  \begin{itemize}
  \item First take the pullback of $f$ and $g$.
  \item Then take the pushout of that pullback.
  \end{itemize}

  \begin{equation*}
    \begin{tikzcd}[ampersand replacement=\&]
      \phantom{A\times_X B} \arrow[d,phantom,swap,"\phantom{\pi_1}"] \arrow[r,phantom,"\phantom{\pi_2}"] \& B \arrow[d,phantom,"\phantom{\inr}"] \arrow[ddr,bend left=15,"g"] \\
      A \arrow[r,phantom,"\phantom{\inl}"] \arrow[drr,bend right=15,swap,"f"] \& \phantom{A\ast_X B} \arrow[dr,phantom,"\phantom{f\ast g}" description] \\
      \& \& X
    \end{tikzcd}
  \end{equation*}
\end{frame}

\begin{frame}
  \frametitle{The fiberwise join of two maps}
  Consider two maps $f:A\to X$ and $g:B\to X$. Then we can
  \begin{itemize}
  \item First take the pullback of $f$ and $g$.
  \item Then take the pushout of that pullback.
  \end{itemize}

  \begin{equation*}
    \begin{tikzcd}[ampersand replacement=\&]
      A\times_X B \arrow[d,swap,"\pi_1"] \arrow[r,"\pi_2"] \& B \arrow[d,phantom,"\phantom{\inr}"] \arrow[ddr,bend left=15,"g"] \\
      A \arrow[r,phantom,"\phantom{\inl}"] \arrow[drr,bend right=15,swap,"f"] \& \phantom{A\ast_X B} \arrow[dr,phantom,"\phantom{f\ast g}" description] \\
      \& \& X
    \end{tikzcd}
  \end{equation*}
\end{frame}

\begin{frame}
  \frametitle{The fiberwise join of two maps}
  Consider two maps $f:A\to X$ and $g:B\to X$. Then we can
  \begin{itemize}
  \item First take the pullback of $f$ and $g$.
  \item Then take the pushout of that pullback.
  \end{itemize}

  \begin{equation*}
    \begin{tikzcd}[ampersand replacement=\&]
      A\times_X B \arrow[d,swap,"\pi_1"] \arrow[r,"\pi_2"] \& B \arrow[d,"\inr"] \arrow[ddr,bend left=15,"g"] \\
      A \arrow[r,"\inl"] \arrow[drr,bend right=15,swap,"f"] \& A\ast_X B \arrow[dr,dotted,"f\ast g" description] \\
      \& \& X
    \end{tikzcd}
  \end{equation*}
\end{frame}

\begin{frame}
  \begin{theorem}
    For any $x:X$, we have an equivalence
    \begin{equation*}
      \fib_{f\ast g}(x)\simeq \fib_f(x)\ast\fib_g(x)
    \end{equation*}
  \end{theorem}

  \begin{proof}
  \begin{equation*}
    \begin{tikzcd}[ampersand replacement=\&]
      \& \fib_f(x)\times\fib_g(x) \arrow[dl] \arrow[dr] \arrow[d] \\
      \fib_f(x) \arrow[d] \& A\times_X B \arrow[dl] \arrow[dr] \& \fib_g(x) \arrow[dl,crossing over] \arrow[d] \\
      A \arrow[dr] \& \unit \arrow[from=ul,crossing over] \arrow[d] \& B \arrow[dl] \\
      \& X
    \end{tikzcd}
  \end{equation*}
  All the squares in this cube are pullback squares.\\[10em]    
  \end{proof}
\end{frame}

\begin{frame}
  \frametitle{Proof (cont)}
  By the descent theorem it therefore follows that the square
  \begin{equation*}
    \begin{tikzcd}[ampersand replacement=\&]
      \fib_f(x)\ast\fib_g(x) \arrow[d] \arrow[r] \& \unit \arrow[d,"x"] \\
      A\ast_X B \arrow[r,swap,"f\ast g"] \& X
    \end{tikzcd}
  \end{equation*}
  is a pullback square. In other words, the fiber of $f\ast g$ at $x$ is $\fib_f(x)\ast\fib_g(x)$.\hfill$\square$
\end{frame}

\begin{frame}
  \begin{definition}
    We define $\RP{n}$ equipped with maps
    \begin{equation*}
      \gamma_n:\RP{n}\to\UU_\bool
    \end{equation*}
    recursively by
    \begin{itemize}
    \item $\RP{0}:=\unit$ and $\gamma_0:\RP{0}\to\UU_\bool$ the base point inclusion, and
    \item $\RP{n+1}:= \RP{n}\ast_{\UU_\bool}\unit$ and $\gamma_{n+1}:=\gamma_n\ast\gamma_0$. 
    \end{itemize}
  \end{definition}
\end{frame}

\begin{frame}
  \phantom{s}\\[3em]
  \begin{theorem}
    For each $n:\N$, we have an equivalence
    \begin{equation*}
      \Big(\sum_{(x:\RP{n})}\gamma_n(x)\Big)\simeq \sphere{n}.
    \end{equation*}
  \end{theorem}

  \begin{proof}
    First, note that we have equivalences
    \begin{equation*}
      \fib_{\gamma_{n+1}}(\bool)\simeq \fib_{\gamma_n}(\bool)\ast \bool
    \end{equation*}
    it follows by recursion that
    \begin{equation*}
      \fib_{\gamma_n}(\bool)\simeq \sphere{n}.
    \end{equation*}
    \\[10em]
  \end{proof}
\end{frame}

\begin{frame}
  \frametitle{Proof (cont)}
      Furthermore, by the fact that $\unit\to\UU_\bool$ is the universal line bundle, we have a fiber sequence
    \begin{equation*}
      \begin{tikzcd}[ampersand replacement=\&]
        \sum_{(x:\RP{n}}\gamma_n(x) \arrow[r] \& \RP{n} \arrow[r,"\gamma_n"] \& \UU_{\bool}
      \end{tikzcd}
    \end{equation*}
    Therefore we obtain that
    \begin{equation*}
      \Big(\sum_{X:\RP{n}}\gamma_n(x)\Big)\simeq\sphere{n}.
    \end{equation*}
    {}\hfill$\square$

    \begin{corollary}
      We have a fiber sequence $\bool\to\sphere{n}\to\RP{n}$ for each $n:\N$.
    \end{corollary}
\end{frame}

\end{document}